\begin{titlepage}
  \begin{center}

  {\Huge VERONICA\_AXI\_BAREMETAL}

  \vspace{25mm}

  \includegraphics[width=0.90\textwidth,height=\textheight,keepaspectratio]{img/AFRL.png}

  \vspace{25mm}

  \today

  \vspace{15mm}

  {\Large Jay Convertino}

  \end{center}
\end{titlepage}

\tableofcontents

\newpage

\section{Usage}

\subsection{Introduction}

\par
The Veronica AXI Baremetal is a base Vexriscv system. This version of the Vexriscv core emulates the E31 core in its mapping of features.
Difference is that this core can generate a MMU and non-PMP enabled versions. The JTAG configuration can also be changed between the Xilinx BSCANE, or
JTAG IO for external devices, or none. Each target tries to use all of the resources of the board using free IP's from the vendor, or open source
IP cores intergrated into fusesoc.

\subsection{Dependencies}

\par
The following are the dependencies of the cores.

\begin{itemize}
  \item fusesoc 2.X
  \item iverilog (simulation)
  \item cocotb (simulation)
\end{itemize}

\subsubsection{fusesoc\_info Depenecies}
\begin{itemize}
\item nexys-a7-100t
	\begin{itemize}
	\item AFRL:utility:digilent\_nexys-a7-100t\_board\_base\_constr:1.0.0
	\item AFRL:utility:digilent\_nexys-a7-100t\_board\_base\_ddr\_cfg:1.0.0
	\item AFRL:utility:vivado\_board\_support\_packages
	\item AD:ethernet:util\_mii\_to\_rmii:1.0.0
	\end{itemize}
\item crosslink-nx\_eval
	\begin{itemize}
	\item AFRL:utility:lattice\_crosslink-nx\_eval\_board\_base:1.0.0
	\end{itemize}
\item dep
	\begin{itemize}
	\item AFRL:utility:helper:1.0.0
	\item AFRL:utility:tcl\_helper\_check:1.0.0
	\item zipcpu:axi\_lite:crossbar:1.0.0
	\item zipcpu:axi:crossbar:1.0.0
	\item zipcpu:axi:sdio:1.0.0
	\item zipcpu:axi:axixclk:1.0.0
	\end{itemize}
\item dep\_uc\_jtag\_io
	\begin{itemize}
	\item spinalhdl:cpu:veronica\_axi\_jtag\_io:1.0.0
	\end{itemize}
\item dep\_uc\_secure\_jtag\_io
	\begin{itemize}
	\item spinalhdl:cpu:veronica\_axi\_secure\_jtag\_io:1.0.0
	\end{itemize}
\item dep\_uc\_jtag\_bscane
	\begin{itemize}
	\item spinalhdl:cpu:veronica\_axi\_jtag\_xilinx\_bscane:1.0.0
	\end{itemize}
\item dep\_uc\_secure\_bscane
	\begin{itemize}
	\item spinalhdl:cpu:veronica\_axi\_secure\_jtag\_xilinx\_bscane:1.0.0
	\end{itemize}
\end{itemize}


\section{Architecture}
\par
The project contains four wrappers

\begin{itemize}
  \item \textbf{system\_wrapper} Contains the top level project module and contains system\_ps\_wrapper.
  \item \textbf{system\_ps\_wrapper} Contains the processor system IP wrappers.
\end{itemize}

\par

Please see \ref{Module Documentation} for more information per target.

\section{Building}

\par
The all Veronica AXI Baremetal project source files are written in Verilog 2001. They should synthesize in any modern FPGA software. The core comes as a fusesoc packaged core and can be
included in any other core. Be sure to make sure you have meet the dependencies listed in the previous section.

\subsection{fusesoc}
\par
Fusesoc is a system for building FPGA software without relying on the internal project management of the tool. Avoiding vendor lock in to Vivado or Quartus.
These cores, when included in a project, can be easily integrated and targets created based upon the end developer needs. The core by itself is not a part of
a system and should be integrated into a fusesoc based system. Simulations are setup to use fusesoc and are a part of its targets.

\subsection{Source Files}

\subsubsection{fusesoc\_info File List}
\begin{itemize}
\item nexys-a7-100t
	\begin{itemize}
	\item {'nexys-a7-100t/system\_wrapper.v': {'file\_type': 'verilogSource'}}
	\item {'nexys-a7-100t/system\_constr.tcl': {'file\_type': 'SDC'}}
	\item {'nexys-a7-100t/system\_gen\_ps.tcl': {'file\_type': 'tclSource'}}
	\item {'nexys-a7-100t/system\_ps\_wrapper.v': {'file\_type': 'verilogSource'}}
	\item {'nexys-a7-100t/system\_gen.tcl': {'file\_type': 'tclSource'}}
	\end{itemize}
\item crosslink-nx\_eval
	\begin{itemize}
	\item {'crosslink-nx\_eval/system\_constr.pdc': {'file\_type': 'PDC'}}
	\item {'crosslink-nx\_eval/system\_wrapper.v': {'file\_type': 'verilogSource'}}
	\end{itemize}
\item dep\_uc\_jtag\_io
	\begin{itemize}
	\item {'nexys-a7-100t/system\_define.tcl': {'file\_type': 'tclSource'}}
	\end{itemize}
\item dep\_uc\_secure\_jtag\_io
	\begin{itemize}
	\item {'nexys-a7-100t/system\_define.tcl': {'file\_type': 'tclSource'}}
	\end{itemize}
\end{itemize}


\subsection{Targets} \label{targets}

\subsubsection{fusesoc\_info Targets}
\begin{itemize}
\item nexys-a7-100t
	\begin{itemize}
	\item[$\space$] Info: Base for nexys-a7-100t digilent development board builds, do not use.
	\end{itemize}
\item nexys-a7-100t\_uc\_secure\_jtag\_io
	\begin{itemize}
	\item[$\space$] Info: Build for nexys-a7-100t digilent development board with PMP enabled Veronica RISCV.
	\end{itemize}
\item nexys-a7-100t\_uc\_jtag\_io
	\begin{itemize}
	\item[$\space$] Info: Build for nexys-a7-100t digilent development board with standard Veronica RISCV.
	\end{itemize}
\item nexys-a7-100t\_uc\_secure\_jtag\_bscane
	\begin{itemize}
	\item[$\space$] Info: Build for nexys-a7-100t digilent development board with PMP enabled Veronica RISCV.
	\end{itemize}
\item nexys-a7-100t\_uc\_jtag\_bscane
	\begin{itemize}
	\item[$\space$] Info: Build for nexys-a7-100t digilent development board with standard Veronica RISCV.
	\end{itemize}
\end{itemize}


\subsection{Directory Guide}

\par
Below highlights important folders from the root of the directory.

\begin{enumerate}
  \item \textbf{docs} Contains all documentation related to this project.
    \begin{itemize}
      \item \textbf{manual} Contains user manual and github page that are generated from the latex sources.
    \end{itemize}
  \item \textbf{nexys-a7-100t} Contains source files for nexys-a7-100t
  \item \textbf{crosslink-nx\_eval} Contains source file for crosslink-nx\_eval, a future target.
\end{enumerate}

\newpage

\section{Simulation}
\par
There is no simulation at the moment. Maybe a future addition?

\newpage

\section{Module Documentation} \label{Module Documentation}

\par
There project has multiple modules. The targets are the top system wrappers.

\begin{itemize}
\item \textbf{nexys-a7-100t}
\item \textbf{crosslink-nx\_eval}
\end{itemize}
The next sections document the module in great detail.

